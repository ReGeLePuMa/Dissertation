% vim: set tw=78 sts=2 sw=2 ts=8 aw et ai:

My tool is still a work in progress and there are several features that are missing or need improvement. 
I have created a comparison table between my tool and Slim in order to highlight the differences and similarities between the two tools.

\begin{table}[htbp]
    \centering
    \caption{Comparison between Dockerminimizer and Slim}
    \label{tab:comparison_dockerslim}
    \begin{tabular}{|l|p{5.5cm}|p{5.5cm}|}
        \hline
        \textbf{Feature} & \textbf{Dockerminimizer} & \textbf{Slim} \\ \hline
        \textbf{Implementation Language} & Go & Go \\ \hline
        \textbf{Analysis Technique} & Static Analysis (`ldd`) \newline Dynamic Analysis (`strace`) \newline Binary Search (Brute Force) & Dynamic Analysis (Sensors/Probes) \newline Static Metadata Analysis \\ \hline
        \textbf{Output Format} & Minimized \texttt{Dockerfile} + \newline Filesystem Tarball & Minified Docker Image (`.slim`) \newline (Opaque Binary) \\ \hline
        \textbf{Fallback Mechanism} & \textbf{Binary Search} \newline (Automated recovery of missing files) & \textbf{Manual Configuration} \newline (User must manually include paths) \\ \hline
        \textbf{Base Image} & \texttt{FROM scratch} & Custom Minified Distro \\ \hline
        \textbf{Transparency} & \textbf{High} (User can inspect/edit the resulting Dockerfile) & \textbf{Low} (Black-box image generation) \\ \hline
        \textbf{Additional Features} & \textbf{None} & \textbf{Full Suite} \newline (\texttt{xray}, \texttt{lint}, \texttt{debug}, \texttt{merge}, \texttt{registry}, \texttt{vulnerability}) \\ \hline
    \end{tabular}
\end{table}

The main advantage of my tool is the transparency of the resulting Dockerfile, which allows users to inspect and modify the minimized image as needed and the
automated binary search procedure that ensures that the minimization process terminates successfully even in cases where dynamic analysis fails.
However, in order to ensure the viability of the tool, I need to either implement some of the additional features that Slim offers or improve on 
the minimization process in order to achieve better results than Slim.

I have also identified several areas for further work, including:
\begin{itemize}
    \item \textbf{Dynamic Analysis Improvements}: In addition to \textit{strace} (which uses ptrace), I could explore other dynamic analysis techniques such as eBPF or systemtap to capture more comprehensive runtime information and potentially improve the accuracy of the minimization process.
    \item \textbf{Binary Search Optimization}: The current brute-force binary search approach can be time-consuming, especially for larger images. I could investigate more efficient search algorithms or heuristics to reduce the number of iterations needed to find the minimal set of files.
    \item \textbf{Security Analysis Integration}: Integrating security analysis tools to identify and remove vulnerable dependencies could enhance the security posture of the minimized images.
    \item \textbf{User Interface Enhancements}: Developing a more user-friendly interface, such as a web dashboard or CLI improvements, could make the tool more accessible to a wider range of users.
    \item \textbf{Support for Additional Container Formats}: Extending support to other container formats (e.g., OCI images) could broaden the applicability of the tool in different container ecosystems.
    \item \textbf{Application Path Analysis}: Implementing a more sophisticated application path analysis to better understand the dependencies and interactions between files could lead to more effective minimization.
\end{itemize}
